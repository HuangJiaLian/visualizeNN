\documentclass{standalone}
\usepackage{tikz}
\begin{document}
\pagestyle{empty}
\def\layersep{3cm}
\def\nodeinlayersep{1.5cm}
\begin{tikzpicture}[
   shorten >=1pt,->,
   draw=black!50,
    node distance=\layersep,
    every pin edge/.style={<-,shorten <=1pt},
    neuron/.style={circle,fill=black!25,minimum size=17pt,inner sep=0pt},
    input neuron/.style={neuron, fill=green!50},
    output neuron/.style={neuron, fill=red!50},
    hidden neuron/.style={neuron, fill=blue!50},
    annot/.style={text width=4em, text centered}
]
    % Draw the input layer nodes
    \foreach \name / \y in {1,...,3}
    % This is the same as writing \foreach \name / \y in {1/1,2/2,3/3,4/4}
    \node[input neuron, pin=left:Input \#\y] (I-\name) at (0,-\y-2.5) {};  %%% <-- MODIFIED
    % set number of hidden layers
    \newcommand\Nhidden{2}

    % Draw the hidden layer nodes
    \foreach \N in {1,...,\Nhidden} {
       \foreach \y in {1,...,5} { %%% MODIFIED (1,...,12 -> 1,...,5, and the next five lines)
         \ifnum \y=4
           \node at (\N*\layersep,-\y*\nodeinlayersep) {$\vdots$};
         \else
           \node[hidden neuron] (H\N-\y) at (\N*\layersep,-\y*\nodeinlayersep ) {$\frac{1}{1+e^{-x}}$};
         \fi
       }
    \node[annot,above of=H\N-1, node distance=1cm] (hl\N) {Hidden layer \N};
    }
    % Draw the output layer node
    \node[output neuron,pin={[pin edge={->}]right:Output}, right of=H\Nhidden-3] (O) {}; %%% <-- MODIFIED (from H\Nhidden-6 to H\Nhidden-3) 
    % Connect every node in the input layer with every node in the
    % hidden layer.
    \foreach \source in {1,...,3}
        \foreach \dest in {1,...,3,5} %%% <-- MODIFIED (1,...,12 -> 1...,3,5)
            \path (I-\source) edge (H1-\dest);
    % connect all hidden stuff
    \foreach [remember=\N as \lastN (initially 1)] \N in {2,...,\Nhidden}
       \foreach \source in {1,...,3,5} %%% <-- MODIFIED (1,...,12 -> 1...,3,5)
           \foreach \dest in {1,...,3,5} %%% <-- MODIFIED (1,...,12 -> 1...,3,5)
               \path (H\lastN-\source) edge (H\N-\dest);
    % Connect every node in the hidden layer with the output layer
    \foreach \source in {1,...,3,5} %%% <-- MODIFIED (1,...,12 -> 1...,3,5)
        \path (H\Nhidden-\source) edge (O);
    % Annotate the layers
    \node[annot,left of=hl1] {Input layer};
    \node[annot,right of=hl\Nhidden] {Output layer};
\end{tikzpicture}
% End of code
\end{document}